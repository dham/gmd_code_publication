\documentclass[a4paper,11pt]{article}

\usepackage[margin=2.5cm]{geometry}
\usepackage{mathpazo}
\usepackage{hyperref}
\usepackage[doi=false,style=authoryear-comp,dashed=false,firstinits=true]{biblatex}
\bibliography{bibliography}

\begin{document}

\title{Nullius in verba: why publishing source code is an essential part of geoscientific
  model development}

\maketitle

Geoscientific Model Development has policies which attempt to ensure that
the source code for the model developments we publish is publicly
available. Why have these policies? The answer to this question is
important not only to justify the responsibilities that it places on
authors, but also in informing what qualifies as good practice in the
publication of source code and data.

Open science and open data are hot topics about which much has been written,
including \cite{Brewer2017}\ explaining and justifying the AGU's open data
policy. Here we focus on the particular issues that arise when that data is
model code.

The short version of the argument is given by the motto of the Royal
Society: ``nullius in verba''. Scientists who publish a result without
publishing the calculations undertaken to achieve that result are requiring
the world to take them at their word, which is inimical to the scientific
method, and the only really effective way to publish the calculations behind
a model is to publish the source code.


\section{Publishing computational science implies publishing code}

\subsection{The correctness of geoscientific models demands code publication}

Geoscientific models exist to enable computational experiments to be
conducted. Those experiments take mathematical systems which are believed to
represent important features of some geoscientific system, and calculate out
the predicted behaviour of that mathematical system in the hope of gaining
insight into the physical one. The hypothesis that such experiments are
testing is:
\begin{enumerate}
\addtocounter{enumi}{-1}
\item This mathematical system captures sufficient features
of the physical system that useful insights about the physical system can be
dawn from it.
\end{enumerate}
Where extensive experiments testing this hypothesis have
been established, one might describe the mathematical system as, in some
sense, a good model for the physical system.

However establishing this evidence places rather strong demands on the
modelling process. In particular there are a number of sub-hypotheses on
which hypothesis 0 depends:
\begin{enumerate}
\item The model code faithfully represents the mathematical specification of
  the model.
\item The computation conducted used the code, input data and configuration intended.
\item The analysis of the model output was itself the correct process.
\end{enumerate}
Publication of source code most directly addresses hypothesis 1, though
hypotheses 2 and 3 are relevant to the best practice in code availability,
and we will return to these in that context.

Let's adopt a simple running example which may serve to illustrate the issue
at hand. Suppose a scientist develops a new advection scheme, and implements
it in an atmospheric dynamics model. They publish a paper describing the new
scheme, and present the results of both idealised verification experiments,
and realistic simulations showing an improvement in forecast skill. The code
is not released. Surely, though, this is a good contribution to modelling
science and valuable to the users of modelling results.

In fact, the reader doesn't know this. Even assuming dilligence and honesty
on the part of the author, there is simply not enough information to know
what the scientists have done and therefore what conclusions could be
drawn. In particular this approach suffers from two key flaws: bugs and the
under-specification of the mathematical model. Indeed, these are both
examples of hypothesis 1 failures, which differ in the source of the
failure.

\subsubsection{Bugs}

Bugs occur in essentially all non-trivial code. Good software engineering
practices such as code review and verification experiments can help reduce
their incidence but can never provably eliminate them. For example
undesirable behaviour may only occur in regimes that were not exercised by
the test suite. The code may therefore unintentionally fail to implement the
mathematics described and this may not be apparent from the tests run so
far.

\subsubsection{Underspecification of the model}

Space in papers is limited, as is the ability of readers to absorb
information and the time authors can spend preparing publications. This
means that even very simple models are significantly underspecified in the
paper. At the more complex end, it is not uncommon for an entire general
circulation model to be described in a 20 page paper (including verification
and some evaluation). This results in all sorts of details being
omitted. Papers frequently omit boundary conditions, or only deal with
simplified cases. More seriously, model code frequently contains fudges and
short-cuts which are neglected in the documentation or paper. For example
perhaps the advection scheme is provably bounded in exact arithmetic, but
roundoff error causes tiny negative values which the model code flushes to
zero to prevent some parametrisation from blowing up. Recording every
implementation detail in the paper would be both impractical and would
undermine the paper as a mechanism for communicating the core ideas behind a
model in a manner intelligible to readers. 

Both bugs and underspecification result in model code whose behaviour
differs from the mathematical system that the author and/or reader believes
is being run. Suppose a reader finds a surprising result in a published
simulation using this model. What should they do? Assuming enough resources
the reader could attempt to reimplement the model: except that
underspecification prevents this. Even if they achieve this, or employ
one or more different models of the same physical system, the most they will
be able to achieve is to conclude that another model fails to reproduce the
original result. Even given two codes attempting to implement the same algorithm, there
is no guarantee that the results will be the same, and without source code
it is impossible to establish which is correct.

\subsection{Source code is necessary but not sufficient}




\section{Journals and archives}

A possible response to the issues caused by lack of access to the source
code is to go back to the authors and ask them to provide code or otherwise
assist in investigating a surprising result. This approach is strongly
dependent on the authors being willing and able to assist. The latter is a
particularly difficult problem: if code has not been curated properly it may
not be possible to recover the exact version used in a paper; the scientist
who did the work may have moved on; or the resources (human or
computational) required to assist may not be available.

All of this undermines the role of a journal as a persistant, public, and
definitive archive of scientific results. In order for journals to fulfill
this function, it must be possible for readers to trace the provenance of
the results presented without further assistance by the authors. For source
code (and other data) to fulfill this role, the data archive must be as
persistent, public and definitive as the journal itself. Relatively small
models and data sets can often be uploaded as supplements to the model and
stored alongside the paper itself. However this approach is impractical for
larger data sets. It is also not always clear that the very high standards
of long-term preservation that journal publishers provide for articles are
also applied to supplements. 

\subsection{Websites and cloud services}

An alternative authors frequently propose is to self-archive the model and data
on a website, or on an online service such as GitHub, GitLab, or Bitbucket. These solutions lack
the required persistence: if the project ends or decides to move to a
different hosting platform, the data will disappear from the location given
in the paper. The mere fact that the author can unilaterally delete the data
is incompatible with the requirement that journal publication be persistent
and definitive.

This is absolutely not to say that model developers should avoid GitHub and
similar services. On the contrary: they are excellent development
platforms which provide revision control, and support code review,
distribution, and integration with continuous integration
systems. Geoscientific model developers should absolutely use services like
GitHub. However when it comes to publishing the exact version of the source
code used in a model, a more persistent solution is required.

\subsection{Third party archives}

A far better solution for source code archiving is a suitable dedicated
third party archive. The persistence and definitive character of such an
archive depends on the institutional arrangements underpinning it. For
source code, Zenodo provides a good option. It is institutionally
underpinned by CERN and provides integration with GitHub which makes the
archiving of git repositories particularly easy. 



\section{Untouched by human hand}

\section{Licences and lawyers}


\end{document}